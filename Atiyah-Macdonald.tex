\documentclass{note}

\begin{document}

\chapter{Rings and Ideals}

\section*{Nilradical and Jacobson Radical}

\Page{5}

\Topic{Minimal prime ideals of a ring}

\textit{Proposition 1.8} shows that the nilradical of a ring is the
intersection of all prime ideals of the ring. But the nilradical itself is not
necessarily a prime ideal, as there may exist zero-divisors other than
nilponent elements. One might wonder what the ``minimal'' prime ideals of a
ring look like, and might think of the set of all zero-divisors as a candidate.
Unfortunately, it might not even be an ideal, as shown in the following
example.
\begin{example*}
  The set of zero-divisors of $\mathbb{Z}_6$, $D =
    \set{\overline{0}}{\overline{2}}{\overline{3}}$, is not an ideal, as
  $\overline{2}+\overline{3}=\overline{5} \notin D$. In fact the ideal generated
  by $D$ is the entire ring.
\end{example*}
However, we can roughly describe the ``minimal'' prime ideals of a ring $A$.
\begin{proposition*}
  Let $D$ be the set of zero-divisors of $A$. Then any largest ideal contained in
  $D$ (ordered by set inclusion) is prime.
\end{proposition*}
\begin{proof}
  By Zorn lemma, such an ideal exists. Let $I$ be such an ideal. Let $x,y\notin
    I$. Then the ideals $(x)+I$, $(y)+I$ are not contained in $D$. Hence there
  exist $a,b\notin D$ such that $$a\in (x)+I, \quad b\in(y)+I.$$ It follows
  $ab\in (xy) + I$. But we know $ab\notin D$, for if $abc=0$, then $bc=0$, then
  $c=0$, thus $ab$ is not a zero-divisor. It then follows $xy\notin I$, because
  otherwise $ab\in (xy) + I = I \subset D$.
\end{proof}

\section*{Exercises}

\setcounter{chapter}{9}

\chapter{Completions}

\section*{Topologies and Completions}

\Page{101 - 102}

\Topic{Lemma 10.1, topology on $G/H$.}

The topology on the quotient group $G/H$ of a topological abelian group $G$ is
just the quotient topology. What is nontrivial and also needed in the lemma
(statement iii)) is that the addition operation is compatible with this
topology.

\begin{lemma*}
  The addition map $G/H \times G/H \to G/H$ defined by $(\bar{x},\bar{y}) \mapsto
    \bar{x}+\bar{y}$ is continuous under the quotient topology.
\end{lemma*}

\begin{proof}
  Let $\overline{W}$ be any open set in $G/H$ and $W$ be its pull back under the
  natural morphism $\pi\colon G \to G/H$. Then $W$ is open. Let $\overline{x},
    \overline{y} \in G/H$ be any elements such that $\overline{x} + \overline{y}
    \in \overline{W}$ with some $x, y \in G$ satisfying $\pi(x) = \overline{x}$ and
  $\pi(y) = \overline{y}$. Then $\pi(x + y) = \pi(x) + \pi(y) \in \overline{W}$,
  implying $x + y$ is in $\overline{W}$'s preimage $W$. By the continuity of the
  addition map $\varphi\colon G \times G \to G$, there exist open sets $U,V
    \subset G$ such that $x\in U$, $y\in V$ and $U\times V \subset
    \varphi^{-1}(W)$. Consider $\overline{U} = \setbuilder{\pi(u)}{u \in U}$. It is
  evident that $\pi^{-1}\left(\overline{U}\right) = \setbuilder{u+h}{u \in U, h
      \in H} = \bigcup_{h\in H}(h+U)$ is open, so $\overline{U}$ is open in $G/H$.
  Define $\overline{V}$ similarly and we get
  $\overline\varphi\left(\overline{U}\times\overline{V}\right) =
    \setbuilder{\pi(u)+\pi(v)}{u\in U, v\in V} = \setbuilder{\pi(u+v)}{u\in U, v\in
      V} = \pi\left(\varphi\left(U\times V\right)\right) \subset \pi(W) =
    \overline{W}$. Therefore $\overline{U} \times \overline{V}$ is an open
  neighborhood of $\left(\overline{x}, \overline{y}\right)$ contained in the
  preimage of $\overline{W}$ under the addition map.
\end{proof}

\begin{remark*}
  Note $H$ is only required to be a subgroup but not necessary closed.
\end{remark*}

A direct way of proving statement iii) without using the structural setup above
is as follows.

\begin{proof}[Proof of lemma 10.1.~iii)] For any
  $\bar{x} \neq \bar{y} \in G/H$, we know that $x - y \notin H$. Let
  the continuous map $\varphi\colon G\times G \to G$ be defined by $(x,y) \mapsto
    x - y$. Then $(x,y) \in G\times G - \varphi^{-1}(H)$, which is open. Therefore
  there exist open sets $U,V \subset G$ such that $x\in U$, $y\in V$ and
  $(U\times V) \cap \varphi^{-1}(H) = \varnothing$. $\overline{U}$ and
  $\overline{V}$ as defined in the lemma above are open neighborhoods of
  $\bar{x}$ and $\bar{y}$. We claim that $\overline{U} \cap \overline{V} =
    \varnothing$. In fact, for any $u\in U$ and $v\in V$, $\bar{u} = \bar{v}$
  implies $u - v \in H$, contradicting the fact that $(U\times V) \cap
    \varphi^{-1}(H) = \varnothing$.
\end{proof}

\section*{Graded Rings and Modules}

\Page{107}

\Topic{Lemma 10.8, $A^*$ and $M^*$.}

Lemma 10.8 uses the following well-known result (proposition 6.5, \page{76}).
\begin{proposition*}
  Let $A$ be a Noetherian ring, $M$ a finitely generated $A$-module. Then $M$ is
  a Noetherian module.
\end{proposition*}
Note this is different from the Hilbert basis theorem (theorem 7.5, \page{81}):
\begin{theorem*}
  If $A$ is Noetherian, then the polynomial ring $\polynomials{A}{x}$ is
  Noetherian.
\end{theorem*}
\end{document}
