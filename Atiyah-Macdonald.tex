\documentclass{note}



\begin{document}

\section*{TODO:}

\begin{itemize}
\item Lemma 10.2 uses (2.10).
\end{itemize}


\chapter{Rings and Ideals}

\section*{Exercises}
\setcounter{chapter}{9}


\chapter{Completions}

\section*{Topologies and Completions}
\Page{101 - 102}
\subsubsection*{Lemma 10.1, topology on $G/H$.}

The topology on the quotient group $G/H$ of a topological abelian
group $G$ is just the quotient topology. What is nontrivial and also
needed in the lemma (statement iii)) is that the addition operation is
compatible with this topology.

\begin{lemma*}
  The addition map $G/H \times G/H \to G/H$ defined by
  $(\bar{x},\bar{y}) \mapsto \bar{x}+\bar{y}$ is continuous under the
  quotient topology.
\end{lemma*}

% \begin{proof}
%   Let $W$ be any open set of $G/H$. If $\pi\colon G \to G/H$ is the
%   natural morphism, then $\pi^{-1}(W)$ is open in $G$. Let
%   $x,y \in G$ be any elements such that $x + y \in \pi^{-1}(W)$. By
%   the continuity of the addition map
%   $\varphi\colon G \times G \to G$, there exist open sets
%   $U,V \subset G$ such that $x\in U$, $y\in V$ and
%   $U\times V \subset \varphi^{-1}\left(\pi^{-1}(W)\right)$. Let
%   $\overline{U} = \{\bar{u} \mid u \in U\}$, then it is evident that
%   $\pi^{-1}\left(\overline{U}\right) = \{u+h \mid u \in U, h \in H\}
%   = \bigcup_{h\in H}(h+U)$, which is open. Define $\overline{V}$
%   similarly and we get for any
%   $(\bar{u},\bar{v}) \in \overline{U}\times\overline{V}$ where
%   $u \in U$ and $v \in V$, that $u + v \in \pi^{-1}(W)$, or
%   $\bar{u} + \bar{v} \in W$. Therefore
%   $\overline{U} \times \overline{V}$ is an open neighborhood of
%   $(\bar{x},\bar{y})$ that is contained in $W$.
% \end{proof}

\begin{proof}
  Let $\overline{W}$ be any open set of $G/H$ whose pull back $W$
  under the natural morphism $\pi\colon G \to G/H$ is open. Let
  $x,y \in G$ be any elements such that $x + y \in W$. By the
  continuity of the addition map $\varphi\colon G \times G \to G$,
  there exist open sets $U,V \subset G$ such that $x\in U$, $y\in V$
  and $U\times V \subset \varphi^{-1}(W)$. Let
  $\overline{U} = \{\bar{u} \mid u \in U\}$, then it is evident that
  $\pi^{-1}\left(\overline{U}\right) = \{u+h \mid u \in U, h \in H\} =
  \bigcup_{h\in H}(h+U)$, which is open. Define $\overline{V}$
  similarly and we get for any
  $(\bar{u},\bar{v}) \in \overline{U}\times\overline{V}$ where
  $u \in U$ and $v \in V$, that $u + v \in W$, or
  $\bar{u} + \bar{v} \in \overline{W}$. Therefore
  $\overline{U} \times \overline{V}$ is an open neighborhood of
  $(\bar{x},\bar{y})$ that is contained in the preimage of
  $\overline{W}$ under the addition map.
\end{proof}

\begin{remark*}
  Note $H$ is only required to be a subgroup but not necessary closed.
\end{remark*}

A direct way of proving statement iii) without using the structural
setup above is as follows.

\begin{proof}[Proof of lemma 10.1.~iii)] For any
  $\bar{x} \neq \bar{y} \in G/H$, we know that $x - y \notin H$. Let
  the continuous map $\varphi\colon G\times G \to G$ be defined by
  $(x,y) \mapsto x - y$. Then $(x,y) \in G\times G - \varphi^{-1}(H)$,
  which is open. Therefore there exist open sets $U,V \subset G$ such
  that $x\in U$, $y\in V$ and
  $(U\times V) \cap \varphi^{-1}(H) = \varnothing$. $\overline{U}$ and
  $\overline{V}$ as defined in the lemma above are open neighborhoods
  of $\bar{x}$ and $\bar{y}$. We claim that
  $\overline{U} \cap \overline{V} = \varnothing$. In fact, for any
  $u\in U$ and $v\in V$, $\bar{u} = \bar{v}$ implies
  $u - v \in H$, contradicting the fact that
  $(U\times V) \cap \varphi^{-1}(H) = \varnothing$.  
\end{proof}

\section*{Graded Rings and Modules}
\Page{107}
\subsubsection*{Lemma 10.8, $A^*$ and $M^*$.}
Lemma 10.8 uses the following well-known result (proposition 6.5,
\page{76}).
\begin{proposition*}
  Let $A$ be a Noetherian ring, $M$ a finitely generated
  $A$-module. Then $M$ is a Noetherian module.
\end{proposition*}
Note this is different from the Hilbert basis theorem (theorem 7.5,
\page{81}):
\begin{theorem*}
  If $A$ is Noetherian, then the polynomial ring $A[x]$ is Noetherian.
\end{theorem*}


\end{document}