\documentclass{note}

\usepackage{stmaryrd}
\usepackage{xifthen}
\usepackage{tikz-cd}

\newcommand{\Page}[1]{\subsection*{Page #1}}
\newcommand{\Ex}[1]{\subsubsection*{Exercise #1}}
\newcommand{\Topic}[1]{\subsubsection*{#1}}

\newcommand{\Affine}[1]{\mathbf{A}^{#1}}
\newcommand{\Proj}[1]{\mathbf{P}^{#1}}
\newcommand{\lring}[2][]{\mathcal{O}_{#2\ifthenelse{\isempty{#1}}{}{,#1}}}

\newcommand{\id}{\mathrm{id}}
\newcommand{\union}{\cup}
\newcommand{\closure}{\overline}
\newcommand{\cmpl}[1]{#1^\text{c}}
\newcommand{\chevrons}[1]{\left\langle#1\right\rangle}
\newcommand{\rational}[2]{\chevrons{#1,#2}}
\newcommand{\divides}{\mid}
\newcommand{\ndivide}{\nmid}

\renewcommand{\theenumi}{\alph{enumi}}


\begin{document}

\chapter{Varieties}

\section{Affine Varieties}
\Page{7}
\Ex{1.2}
$Y$ is the zero set of the ideal $(x^2-y,x^3-z)$. One only needs to
show that $A(Y)$ is isomorphic to a polynomial ring in one variable,
from which the rest follows.

\section{Projective Varieties}
\Page{13}
\Ex{2.12}
\begin{enumerate}
\item Since polynomial rings are integral, so is
  $k[y_0,\dots,y_N]/\ideal{a}$, so $\ideal{a}$ is prime. (Need to show
  it is homogeneous?)
\item $\rho_d(\Proj{n})\subset Z(\ideal{a})$ is obvious. In fact for
  any $\rho_d(P)=(M_0(a),\dots,M_N(a))$, and any $f\in\ideal{a}$, we
  have
  $f(\rho_d(P)) = f(M_0(a),\dots,M_N(a)) \xlongequal{\text{by
      definition of }\theta} \theta(f)(a) = 0.$\\
  $Z(\ideal{a})\subset\rho_d(\Proj{n})$ is trivial. Just need to
  construct $\rho_d^{-1}(P)$ from $P\in Z(\alpha)$.
\item One direction is easy. To show $\rho_d^{-1}$ is continuous, one
  needs to look at the open sets $U_i$.
\end{enumerate}

\section{Morphisms}

\Page{20}

\Ex{3.1}
\begin{enumerate}
\item
\item
\item Any non-degenerate linear transformation
  $\mathbf{x} \mapsfrom M\mathbf{x}$ induces an isomorphism between a
  variety and its image. For any conic
  $Ax^2 + By^2 + Cz^2 + Dyz + Ezx + Fxy = 0$ in $\Proj{2}$, we show
  under a suitable transformation it becomes $xy = z^2$.
  \begin{itemize}
  \item If at most one in $A$, $B$, $C$ is nonzero, say $C$,
    dehomogenize with respect to $z$ we get
    \begin{equation*}
      Fxy + Ex + Dy + C = 0.
    \end{equation*}
    Obviously $F \ne 0$ and by change of variables we can bring it to
    \begin{equation*}
      xy = 1.
    \end{equation*}
    That the constant term is not zero is a consequence of the fact
    that the quadratic equation is irreducible. Now homogenize to $z$
    and we get the desired form.
  \item If two or more in $A$, $B$, $C$ are nonzero, say $A$ and
    $B$. Dehomogenize w.r.t.~$z$ to get
    \begin{equation*}
      Ax^2 + Fxy + By^2 + Ex + Dy + C = 0.
    \end{equation*}
    By transformation
    \begin{equation*}
      \begin{bmatrix}
        x \\ y
      \end{bmatrix}
      \mapsfrom
      \begin{bmatrix}
        \sqrt{B} & \sqrt{B} \\
        \sqrt{A} & -\sqrt{A}
      \end{bmatrix}
      \begin{bmatrix}
        x \\ y
      \end{bmatrix}
    \end{equation*}
    the curve is brought to
    $(2AB + F\sqrt{AB})x^2 + (2AB - F\sqrt{AB})y^2 + E'x + D'y + C =
    0$. Since $A,B\ne0$, divide by $\sqrt{AB}$ to get
    $(2\sqrt{AB} + F)x^2 + (2\sqrt{AB} - F)y^2 + E'x + D'y + C' =
    0$. The two quadratic coefficients cannot be both 0, so the
    equation can be converted into either
    \begin{equation*}
      x^2 + y^2 = 1, \quad \text{or} \quad x^2 = y.
    \end{equation*}
    In the second case it is already the desired form. In the first
    case homogenization gives $x^2 + y^2 = z^2$. Write
    $(x + iy)(x - iy) = z^2$ and we are done.
  \end{itemize}
  Now we need to show the curve $Y$ given by $xy = z^2$ is isomorphic
  to $\Proj{1}$. Consider the map $\varphi\colon\Proj{1} \to Y$ given
  by $\varphi((x,y)) = (x^2,y^2,xy)$. It is not hard to check that it
  admits an inverse.
  
\item
\item 
\end{enumerate}

\Page{21}

\Ex{3.3}
\begin{enumerate}
\item Need to varify: well-definedness, compatibility with ring
  structures.
\item
  ``$\implies$'': Trivial.\\
  ``$\impliedby$'': Only need to show $\varphi^{-1}$ is a
  morphism. For any open subset $U$ of $X$ and any regular function
  $f$ on $U$, need to show $f\circ\varphi^{-1}$ is regular on
  $\varphi(U)$. For any point $\varphi(P)$ in $\varphi(U)$, since
  $\varphi_P^*$ is an isomorphism, $f$ viewed as a function in
  $\lring[X]{P}$ has some preimage
  $g\in\lring[Y]{\varphi(P)}$. Because
  $\varphi_P^*(g) = g\circ\varphi$, $f$ and $g\circ\varphi$ agree on
  some open set $V\subset U$. In other words, $f\circ\varphi^{-1} = g$ on
  $\varphi(V)$, proving $f\circ\varphi^{-1}$ is regular at
  $\varphi(P)$.
\item Suppose $\chevrons{V,g}\in\lring[Y]{\varphi(P)}$ and
  $0 = \varphi_P^*(g) = g\circ\varphi$. Then $g\circ\varphi$ vanishes
  in an open neighborhood of $P$ contained in $\varphi^{-1}(V)$, which
  means it vanishes in $\varphi^{-1}(V)\eqqcolon U$. Thus $g$ vanishes
  in $\varphi(U)$. Now the closure of $\varphi(X)$ is contained in
  $\closure{\varphi(U)}\union\cmpl{V}$. Since $\varphi(X)$ is dense in
  $Y$, we have $\closure{\varphi(U)}\union\cmpl{V} = Y$. This is the
  case only if $V\subset\closure{\varphi(U)}$, in which $g$
  vanishes. Thus we conclude $g = 0$ in $\lring[Y]{\varphi(P)}$.
\end{enumerate}

\Ex{3.5}
Follow hints. More details to be added.

\Ex{3.7}
Just prove (b).

\section{Rational Maps}
\Page{24}
\subsubsection*{The composition of dominant rational maps is dominant}
Let $\rational{U}{\varphi_U}$ be dominant rational map from $X$ to $Y$
and $\rational{V}{\psi_V}$ be dominant from $Y$ to $Z$. The
composition is defined where $\psi_V\circ\varphi_U$ is defined. Let
$V'\coloneqq\varphi_U(U)\cap V$. If $W\coloneqq\closure{\psi_V(V')}$
is not $Z$, then obviously $\psi_V^{-1}(W) \subsetneq V$, thus
$\psi_V^{-1}(Z\setminus W)$ is a nonempty open subset of $V$, thus
open in $Y$. Now
$\varphi_U(U) \cap \psi_V^{-1}(Z\setminus W) = \varnothing$ (since
$\varphi_U(U) \cap V \subset V' \subset \psi_V^{-1}(W)$),
contradicting that $\varphi_U(U)$ is dense.

\Page{30}
\Ex{4.4}
\begin{enumerate}
\item By exercise 3.1(c).
\item
\item
\end{enumerate}

\section{Nonsingular Varieties}
\Page{32}
\subsubsection*{Details in the proof of theorem 5.1 about the
  localization properties}
\begin{proposition*}
  Let $A$ be a ring, $\ideal{b}$ an ideal,
  $\ideal{a} \supset \ideal{b}$ a maximal ideal. Let $\mathcal{O}$ be
  the localization of the quotient ring $A/\ideal{b}$ at
  $\ideal{a}/\ideal{b}$, and $\ideal{m}$ the maximal ideal of
  $\mathcal{O}$. Then $\ideal{m} = (\ideal{a})$. The residues fields
  $\mathcal{O}/\ideal{m}$ and $A/\ideal{a}$ are isomorphic. Denote
  this field by $k$. Then $\ideal{m}/\ideal{m}^2$ and
  $\ideal{a}/(\ideal{a}^2+\ideal{b})$ are vector spaces over $k$. We
  have the following isomorphism of vector spaces:
  \begin{equation*}
    \ideal{m}/\ideal{m}^2 \cong \ideal{a}/(\ideal{a}^2+\ideal{b}).
  \end{equation*}
\end{proposition*}

\begin{proof}
  We first show the isomorphism between $\mathcal{O}/\ideal{m}$ and
  $A/\ideal{a}$. Consider the ring morphism
  $\alpha\colon A \to \mathcal{O}/\ideal{m}$ defined via the natural
  morphism $x \mapsto \frac{\bar{x}}{1}+\ideal{m}$, where $\bar{x}$
  denotes the equivalent class of $x$ in $A/\ideal{b}$. Then $\alpha$
  is onto. In fact any element in $\mathcal{O}/\ideal{m}$ has a
  representative $\frac{\bar{x}}{\bar{y}}$, where $x\in A$,
  $y \in A\setminus\ideal{a}$. Since $\ideal{a}$ is maximal, $y$ has
  an inverse $\!\ilmod\ideal{a}$ and we are done. One can also show
  that the kernel of $\alpha$ is $\ideal{a}$. Therefore we obtain the
  isomorphism $A/\ideal{a} \xrightarrow{\sim} \mathcal{O}/\ideal{m}$.

  To show the two vector spaces are isomorphic we consider the linear
  map $\varphi\colon \ideal{a} \to \ideal{m}/\ideal{m}^2$ defined also
  by the natural morphism $x\mapsto\frac{\bar{x}}{1}+\ideal{m}^2$. We
  first show $\varphi$ is onto. Any element in $\ideal{m}/\ideal{m}^2$
  has a representative $\frac{\bar{x}}{\bar{y}}$, where
  $x \in \ideal{a}$, $y \in A\setminus\ideal{a}$. Suppose
  $yz \equiv 1 \ilmod \ideal{a}$, then
  $\bar{y}\bar{z} \equiv 1 \ilmod \ideal{m}$, or
  $\frac{1}{\bar{y}}-\bar{z} \in \ideal{m}$. Therefore
  $\bar{x}\left(\frac{1}{\bar{y}}-\bar{z}\right) \in \ideal{m}^2$
  which means
  $\frac{\bar{x}}{\bar{y}} \equiv \bar{x}\bar{z} \ilmod \ideal{m}^2$,
  and we get
  $\varphi(xz) = \bar{x}\bar{z}+\ideal{m}^2 =
  \frac{\bar{x}}{\bar{y}}+\ideal{m}^2$. We then show that the kernel
  of $\varphi$ is $\ideal{a}^2+\ideal{b}$. For $x \in \ideal{a}$,
  $\varphi(x) = 0$ if and only if
  $\bar{x} = x+\ideal{b} \in \ideal{m}^2 = (\ideal{a}^2)$, or
  $x+\ideal{b} = \frac{h+\ideal{b}}{g+\ideal{b}}$ for some
  $h\in\ideal{a}^2$ and $g\in A\setminus\ideal{a}$. We then get
  $xg - h \in \ideal{b}$. There exists some $f$ such that
  $fg-1\in \ideal{a}$, therefore
  $x = x(1-fg+fg) = x(1-fg) + xfg = x(1-fg) + x(fg - h) + xh \in
  \ideal{a}^2 + \ideal{b} + \ideal{a}^3 = \ideal{a}^2+\ideal{b}$.
\end{proof}

\Page{37}
\Ex{5.5}
\begin{itemize}
\item If $p\ndivide d$ consider $x^d + y^d + z^d = 0$.
\item If $p\divides d$ consider $x^{d-1}y + y^{d-1}z + z^{d-1}x =
  0$. The polynomial is irreducible since $x^{d-1}y + y^{d-1} + x$ is
  (Eisenstein's criterion on integral domains). Further more it is
  nonsingular.
\end{itemize}

\Ex{5.8}
Use the fact that if $g_i$ is a polynomial of degree $d$, and $f_i$ is
homogeneous polynomial
\begin{equation*}
  f_i = x_0^d g_i\left(\frac{x_1}{x_0}, \dots, \frac{x_n}{x_0}\right),
\end{equation*}
then
\begin{equation*}
  \frac{\partial f_i}{\partial x_j}
  = x_0^d\frac{\partial g_i}{\partial x_j}\frac{1}{x_0}
  = x_0^{d-1}\frac{\partial g_i}{\partial x_j}.
\end{equation*}

\section{Nonsingular Curves}
\Page{40}
\subsubsection*{Every localization of a Dedekind domain at a nonzero
  prime ideal is a discrete valuation ring}
Need that noetherian and dimension are also local properties?

\subsubsection*{Closed proper subsets of a curve are finite}
By dimension theory. See
\href{https://math.stackexchange.com/questions/140592/closed-proper-subvarieties-of-curves-are-finite-sets-of-points}
{this answer} on Stack Exchange. A similar argument can be used to
show that a variety of dimension 1 must be a single point.

\Page{44}
\subsubsection*{The uniqueness of $\bar\varphi$ is clear by
  construction}
Suppose $\bar\varphi'$ is another morphism extending $\varphi$. Then
$x_i/x_k\circ\bar\varphi'$ and $x_i/x_k\circ\bar\varphi$ agree on an
open subset of $X$ (for all $i$), thus must be the same regular
function on $X$. In particular,
$x_i/x_k\circ\bar\varphi'(P) = x_i/x_k\circ\bar\varphi(P) =
f_{ik}(P)$, i.e.,
$\dfrac{\bar\varphi'(P)_i}{\bar\varphi'(P)_k} = f_{ik}(P)$, showing
$\bar\varphi(P) = \varphi(P)$.

\subsubsection*{Theorem 6.9 paragraph 2: why not use lemma 6.5 to
  prove local isomorphisms.}
Actually lemma 6.5 only constructs an affine nonsingular curve with
the appropriate affine coordinate ring, without proving any
isomorphisms.

\Page{45}
\subsubsection*{Theorem 6.9 last paragraph.}
A bijective map between two curves (abstract or not) is necessarily a
homeomorphism, because the closed sets of a curve are just finite
sets.

\Page{46}
\subsubsection*{Corollary 6.12}
Denote the functor (i) $\to$ (ii) $\to$ (iii) by $F$ and (iii) $\to$
(i) by $G$. We need to show $GF=\id_{\text{(i)}}$ and
$FG=\id_{\text{(iii)}}$.

For a nonsingular projective curve $X$, let $K$ be its function
field. Then $GF(X) = C_K$. One needs to construct an isomorphism
$\mu_X\colon X\to C_K$. $X$ and $C_K$ are birational since they have
the same function field $K$. Thus there exist open sets $U$ and $V$ in
$X$ and $C_K$ respectively, which are isomorphic. Let
$\mu\colon U\to V$ be the isomorphism and $\nu$ be its inverse. Since
$X$ and $C_K$ are both nonsingular projective curves, by (6.8) one can
uniquely extend $\mu$ and $\nu$ to $\bar\mu\colon X\to C_K$ and
$\bar\nu\colon C_K\to X$. Now $\bar\nu\circ\bar\mu$ is an extension of
$\nu\circ\mu = \id_U$, which on the other hand extends (uniquely) to
$\id_X$. Therefore we get an isomorphism $\bar\mu\colon X\to C_K$.

For any nonsingular projective curves $X_1$ and $X_2$ and dominant
morphism $\varphi\colon X_1\to X_2$. One needs to show the following
diagram commutes.
\begin{equation*}
  \begin{tikzcd}[row sep=huge,column sep=huge]
    X_1 \arrow{r}{\mu_{X_1}} \arrow[swap]{d}{\varphi} & C_{K_1}
    \arrow{d}{GF(\varphi)} \\%
    X_2 \arrow{r}{\mu_{X_2}} & C_{K_2}
  \end{tikzcd}
\end{equation*}
However the same commutative diagram with rational maps in place of
the morphisms exists due to the equivalence of categories in
(4.4). Then by going to extensions we get the desired result.

\Ex{6.1}
\begin{enumerate}
\item $Y$ is isomorphic to an open subset of a nonsingular projective
  curve $X$ by corollary 6.10. Then $X$ and $\Proj1$ are birational,
  and hence isomorphic due to proposition 6.8. Since $Y$ is not
  isomorphic to $\Proj1$ the result is immediate.
\item An open set in $\Affine1$ is
  $\Affine1\setminus\{x_1,\dots,x_n\}$. It is isomorphic to the affine
  curve given by $(x-x_1)\cdots(x-x_n)y=1$ in $\Affine2$.
\item Show that $A(Y)$ is isomorphic to the localization of $k[x]$ at
  the polynomials $x-x_i\colon$
  \begin{equation*}
    \left\{ \frac{f(x)}{\prod_{i=1}^n(x-x_i)^{k_i}} \biggm\vert
      f(x) \in k[x] \right\}.
  \end{equation*}
\end{enumerate}


\Ex{6.2}
\begin{enumerate}
\item If $Y$ were to be singular at some point, then it must be that
  $y = 0$ and $3x^2-1 = 0$, which contradicts $y^2 = x^3 - x$. Since
  $Y$ is nonsingular every localization of $A$ is integrally closed,
  so $A$ is integrally closed.
\item Only need to show that if $f(x) \in k[x]$ vanishes on $Y$, then
  $f(x)$ is the zero polynomial. For every $x\in k$ there is some $P$
  whose first coordinate is $x$, thus $f(x) = 0$ for every $x$ and we
  are done. $A$ is in the integral closure of $k[x]$ since
  $y^2 + x^3 - x = 0$ in $K$. Then using the fact that $A$ is
  integrally closed we get the result.
\item 
\end{enumerate}

\end{document}