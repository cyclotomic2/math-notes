\documentclass[12pt]{note}

\usepackage[margin=1in]{geometry}
\usepackage{amsmath}
\usepackage{xcolor}
\usepackage{fontspec}
\usepackage{titlesec}

\titleformat{\chapter}[display]
    {\LARGE\sffamily}
    {\chaptertitlename\space\thechapter}
    {1em}
    {\fontspec{Arial}\selectfont\bfseries\itshape}
\renewcommand{\thesection}{\thechapter.\arabic{section}}
\titleformat{\section}[block]
    {\Large\bfseries\filcenter}
    {\thesection}
    {1em}
    {\fontspec{Times New Roman}\selectfont\MakeUppercase}

\numberwithin{equation}{chapter}
\makeatletter
\def\tagform@#1{\maketag@@@{[\ignorespaces#1\unskip\@@italiccorr]}}
\makeatother

\newtagform{brackets}{(}{)}
\usetagform{brackets}

\renewcommand{\theenumi}{\alph{enumi}}

\newcommand{\piprod}[2]{
    \frac{1\cdot2\cdot\cdots\cdot #2}{\prn{#1+1}\prn{#1+2}\cdots\prn{#1+#2}}
    \prn{#2+1}^{#1}
}
\newcommand{\re}{\operatorname{Re}}


\begin{document}

\chapter{Riemann's Paper}

\setcounter{section}{2}
\section{The Factorial Function}

\Page{8}

\begin{quotebar}
    On the other hand, it is not difficult to show [use formula (4) below] that the
    limit (3) exists for all values of $s$, real or complex, provided only that the
    denominator is not zero, that is, provided only that $s$ is not a negative
    integer.
\end{quotebar}

Rewrite the right-hand side of (4) as
\begin{equation}
    \prod_{n=1}^N \prn{1+\frac{s}{n}}^{-1} \prn{1+\frac{1}{n}}^{s} = \exp{
        \sum_{n=1}^N \left[s\log\prn{1+\frac{1}{n}} - \log\prn{1+\frac{s}{n}} \right]}.
    \label{eq:rewrite-Pi-prod-form}
\end{equation}
Note the definition of $\log\prn{1+\frac{1}{n}}$ and $\log\prn{1+\frac{s}{n}}$
is unambiguous when $n$ is sufficiently large and $s$ is fixed. According to
\cite[p. 125, Theorem 8]{ahlfors1979complex}, we can rewrite $\log(1+z)$ as
\begin{equation*}
    \log(1+z) = \log(1+0) + \frac{1}{1+0}z + f(z)z^2 = z + f(z)z^2
\end{equation*}
where $f$ is analytic in the domain of $\log(1+z)$ (chosen to be the complex
plain excluding real values $s \leq -1$). Thus
\begin{align}
    \abs{s\log\prn{1+\frac{1}{n}} - \log\prn{1+\frac{s}{n}}}
     & = \abs{ s\left[ \frac{1}{n} + f\prn{\frac{1}{n}}\prn{\frac{1}{n}}^2 \right] -
    \left[ \frac{s}{n} + f\prn{\frac{s}{n}}\prn{\frac{s}{n}}^2 \right] } \nonumber   \\
     & = \abs{ s f\prn{\frac{1}{n}} - s^2 f\prn{\frac{s}{n}} } \frac{1}{n^2}
    \label{eq:Pi-prod-summand-taylor}                                                \\
     & = O\prn{\frac{1}{n^2}} \quad \text{as } n \rightarrow \infty, \nonumber
\end{align}
which concludes that the sum on the right-hand side of
\eqref{eq:rewrite-Pi-prod-form} is absolute convergent.

\begin{quotebar}
    \begin{equation*}
        \Pi(s) = 2^s \Pi \prn{\frac{s}{2}} \Pi \prn{\frac{s - 1}{2}} \pi^{-1/2}
    \end{equation*}
\end{quotebar}

By equation (3) in the book,
\begin{align*}
    \Pi\prn{\frac{s}{2}}\Pi\prn{\frac{s-1}{2}}
     & = \lim_{N\rightarrow\infty} \left[ \piprod{\frac{s}{2}}{N} \right.             \\
     & \hspace{5em} \cdot \left. \piprod{\frac{s-1}{2}}{N} \right]                    \\
     & = \lim_{N\rightarrow\infty} \frac{\prn{1\cdot2\cdot\cdots\cdot N}^2}
    {\frac{1}{2^{2N}}\prn{s+1}\prn{s+2}\cdots\prn{s+2N}}\prn{N+1}^{s-\frac12}         \\
     & = \lim_{N\rightarrow\infty} \frac{2\cdot4\cdot\cdots\cdot 2N}{\prn{s+1}
        \prn{s+2}\cdots\prn{s+2N}} \cdot \frac{1\cdot2\cdot\cdots\cdot N}
    {\frac{1}{2^{N}}}\prn{N+1}^{s-\frac12}                                            \\
     & = \lim_{N\rightarrow\infty} \left[ \piprod{s}{2N} \right.                      \\
     & \hspace{5em} \left. \cdot \frac{1\cdot2\cdot\cdots\cdot N}{\frac{1}{2^{N}}
            1\cdot3\cdot\cdots\cdot (2N-1)} \cdot \frac{\prn{N+1}^{s-\frac12}}
    {\prn{2N+1}^s} \right]                                                            \\
     & = \lim_{N\rightarrow\infty} \left[ \piprod{s}{2N} \right.                      \\
     & \hspace{5em} \left. \cdot \piprod{-\frac12}{N} \cdot \frac{\prn{N+1}^{s}}
    {\prn{2N+1}^s} \right]                                                            \\
     & = \lim_{N\rightarrow\infty} \piprod{s}{2N}                                     \\
     & \hspace{5em} \cdot \lim_{N\rightarrow\infty} \piprod{-\frac12}{N}              \\
     & \hspace{5em} \cdot \lim_{N\rightarrow\infty}\frac{\prn{N+1}^{s}}{\prn{2N+1}^s} \\
     & = \Pi\prn{s} \Pi\prn{-\frac12} \prn{\frac12}^s = \Pi\prn{s} \pi^{\frac12}
    2^{-s},
\end{align*}
which implies
\begin{equation*}
    \Pi\prn{s} = 2^{s} \Pi\prn{\frac{s}{2}} \Pi\prn{\frac{s-1}{2}} \pi^{-\frac12}.
\end{equation*}

\begin{quotebar}
    Using it one can prove that $\Pi(s)$ is an analytic function of the complex
    variable s which has simple poles at $s = -1, -2, -3, \dots$.
\end{quotebar}

We showed above that $\Pi(s)$ is well-defined excluding $s = -1, -2, -3,
    \dots$. Now we show $\Pi(s)$ is analytic using equation
\eqref{eq:Pi-prod-summand-taylor}. According to \cite[p.~176, Theorem
    1]{ahlfors1979complex}, it suffices to show that
\eqref{eq:Pi-prod-summand-taylor} converges uniformly on compact subsets of the
domain of $\Pi(s)$, which is obvious by setting $n$ large enough and observing
that $s$ is bounded on any compact subset (so that $s/n$ is close to $0$ and
the function $s f\prn{1/n} - s^2 f\prn{s/n}$ is bounded).

To show that $\Pi(s)$ has simple poles at $s = -1, -2, -3, \dots$, it suffices
to multiply the product form of $\Pi(s)$ in \eqref{eq:rewrite-Pi-prod-form} by
$(s+n)$ and repeat the same argument.

\setcounter{section}{15}
\section{The Remaining Terms}

\Page{32}

\begin{quotebar}
    Note first that the series
    \[
        \frac{d}{ds} \left[ \frac{\log \Pi(s/2)}{s} \right] = -\sum_{n=1}^\infty
        \frac{d}{ds} \left\{ \frac{\log[1 + (s/2n)]}{s} \right\}
    \]
    converges uniformly in any disk $|s| \leq K$. [For large $n$ the series
            expansion $\log(1 + x) = x - \frac{1}{2}x^2 + \frac{1}{3}x^3 - \cdots$ can be
            used, and the summand on the right contains only terms in $n^{-2}, n^{-3},
                \ldots.$]
\end{quotebar}

By ``the summand on the right'', we do not refer to the equation quoted above,
but to the second equation at the top of the page
\[
    \log \Pi\left(\frac{s}{2}\right) = \sum_{n=1}^\infty \left[ -\log\left(1 +
        \frac{s}{2n}\right) + \frac{s}{2} \log\left(1 + \frac{1}{n}\right)
        \right],
\]
and \cite[p.~176, Theorem 1]{ahlfors1979complex} is used.

\Page{33}

\begin{quotebar}
    Integration by parts puts this in the form
    \begin{align*}
        = & \frac{1}{2\pi i} \frac{x^a}{2n \log x} \cdot \frac{1}{2n \log x}
        \left( \left.\left\{ \frac{d}{dv} \left[ \frac{\log(1 + v + b)}{v + b}
        \right] x^{2nv} \right\}\right|_{v = -ic}^{v = ic} \right.           \\
          & \left. - \int_{-ic}^{ic} \frac{d^2}{dv^2} \left[
            \frac{\log(1 + v + b)}{v + b} \right] x^{2nv} dv \right).
    \end{align*}
    Now $b$ is a real number $0 \leq b \leq a$, the function
    \begin{equation*}
        \frac{d}{dv} \left[ \frac{\log(1 + v + b)}{v + b} \right] = \frac{1}{(v + b)(v
            + b + 1)} - \frac{\log(1 + v + b)}{(v + b)^2}
    \end{equation*}
    is bounded on the imaginary axis, and its derivative is absolutely integrable
    over $(-i\infty,$ $i\infty)$, from which it follows that the modulus of the
    $n$th term of the series on the right side of $(2)$ is at most a constant times
    $n^{-2}$ for all $T$.
\end{quotebar}

Actually, it doesn't suffice that the function is bounded on the imaginary
axis, because $b = a/2n$ is not a constant and changes for each term in the
sum. But luckily we can show a better result. Let $z = v + b$. We note $z$ lies
in the strip $0 \leq \re z \le a$. The function above is then
\begin{equation*}
    f(z) = \frac{1}{z(z + 1)} - \frac{\log(1 + z)}{z^2}.
\end{equation*}
$f(z)$ is analytic on the strip $0 \leq \re z \le a$ except for a singularity
at $z = 0$. However, this singularity is removable because
\begin{align*}
    \lim_{z \to 0} zf(z)
     & = \lim_{z \to 0} \left\{ \frac{1}{z + 1} - \frac{\log(1 + z)}{z}
    \right\} = 1 - \lim_{z \to 0} \frac{\log(1 + z)}{z}                 \\
     & = 1 - \left\{\log(1+z)\right\}'|_{z = 0} = 1 - 1 = 0,
\end{align*}
where we have used \cite[p.~124, Theorem 7]{ahlfors1979complex}. Thus we have
shown that $f(z)$ is analytic on the entire strip. On the other hand, it is not
hard to see that $f(z)$ is bounded for large $\abs{z}$. These two facts imply
that $f(z)$ is bounded on the strip, which is the better result we were to
show.

The derivative of the function is
\begin{align*}
      & - \frac{2v + 2b + 1}{(v + b)^2(v + b + 1)^2}
    - \frac{1}{(v + b)^2 (v + b + 1)} + \frac{2\log(1 + v + b)}{(v + b)^3} \\
    = & - \frac{3v + 3b + 2}{(v + b)^2(v + b + 1)^2}
    + \frac{2\log(1 + v + b)}{(v + b)^3},
\end{align*}
the modulo of which is $\leq O(\abs{v}^{-3} + \log\prn{\abs{v}}\abs{v}^{-3})$
as $v \to \pm i\infty$, which is absolutely integrable.

\setcounter{chapter}{2}
\chapter{Riemann's Main Formula}

\setcounter{section}{1}
\section{Derivation of von Mangoldt's Formula for \texorpdfstring{$\psi(x)$}{ψ(x)}}

\Page{53}

\begin{quotebar}
    The second method of evaluating the integral (1) is as follows. Differentiate
    logarithmically the formula
    \[
        \Pi \left( \frac{s}{2} \right) \pi^{-s/2} (s - 1) \zeta(s) = \xi(0)
        \prod_{\rho} \left( 1 - \frac{s}{\rho} \right)
    \]
    to find
    \begin{align*}
         & \frac{d}{ds} \log \Pi \left( \frac{s}{2} \right) - \frac{1}{2} \log
        \pi + \frac{1}{s - 1} + \frac{\zeta'(s)}{\zeta(s)}                     \\
         & \qquad = \sum_{\rho} \frac{1}{1 - (s/\rho)} \cdot \left(
        -\frac{1}{\rho} \right).
    \end{align*}
    Using the expression of \( \Pi(x) \) as an infinite product [(4) of Section
            1.3], and differentiating termwise then gives
    \begin{align*}
        \tag{6} -\frac{\zeta'(s)}{\zeta(s)} =
         & \frac{1}{s - 1} - \sum_{\rho} \frac{1}{s - \rho}                  \\
         & + \sum_{n=1}^{\infty} \left[ -\frac{1}{s + 2n} + \frac{1}{2} \log
            \left( 1 + \frac{1}{n} \right) \right] - \frac{1}{2} \log \pi.
    \end{align*}

    \vspace{1ex} \centerline{\vdots} \vspace{1em}

    \dots, note first that both of the infinite series in (6) converge uniformly in
    any disk \( |s| \leq K \). (The series in \( n \) converges uniformly because
    \begin{align*}
         & \abs{(s + 2n)^{-1} - \textstyle\frac{1}{2} \log(1 + n^{-1})}      \\
         & \qquad = |(s + 2n)^{-1} - (2n)^{-1} + (2n)^{-1}                   \\
         & \qquad \quad - \textstyle\frac{1}{2}(n^{-1} - \frac{1}{2}n^{-2} +
        \frac{1}{3}n^{-3} - \cdots )|                                        \\
         & \qquad \leq |s(s + 2n)^{-1}(2n)^{-1}| + \textstyle \abs{
        \frac{1}{4}n^{-2} - \frac{1}{6}n^{-3} + \cdots}                      \\
         & \qquad \leq K(2n)^{-2} + n^{-2} \leq \text{const}/n^2
    \end{align*}
    for all sufficiently large \( n \), and the series in \( \rho \) converges
    uniformly because when the terms \( \rho \) and \( 1 - \rho \) are paired
    \begin{align*}
         & |(s - \rho)^{-1} + [s - (1 - \rho)]^{-1}|                          \\
         & \qquad = \left| \left[\prn{s - \frac{1}{2}} - \prn{\rho -
                \frac{1}{2}}\right]^{-1} + \left[\prn{s - \frac{1}{2}} +
        \prn{\rho - \frac{1}{2}}\right]^{-1} \right|                          \\
         & \qquad = \left\lvert \frac{2(s - \frac{1}{2})}{(s - \frac{1}{2})^2
            - (\rho - \frac{1}{2})^2} \right\rvert
        \leq \text{const} \left\lvert \rho - \frac{1}{2} \right\rvert^{-2}
    \end{align*}
    for all sufficiently large \( \rho \) once \( K \) is fixed and because \( \sum
    |\rho - \frac{1}{2}|^{-2} \) converges by the theorem of Section 2.5.) This
    proves that the termwise differentiation by which (6) was obtained is valid.
\end{quotebar}

This uses the following theorem.
\begin{theorem*}
    If a sequence of analytic functions \( f_n
    \) converges pointwise to a function \( f \) in a region \( \Omega \), and if
    \( f_n' \) converges to \( g \) uniformly on compact subsets of \( \Omega \),
    then \( f \) is analytic, \( g = f' \), and \( f_n \) converges to \( f \)
    uniformly on compact subsets of \( \Omega \).
\end{theorem*}

Specifically, the series $\sum_\rho\log\prn{1 - \frac{s}{\rho}}$ converges
pointwise which is proved in section 2.5, and the series of its derivatives
$\sum_\rho\prn{s - \rho}^{-1}$ converges uniformly on any disk $|s| \leq K$ as
shown above, so the conditions of the theorem are met and
\begin{equation*}
    \frac{d}{ds} \sum_\rho\log\prn{1 - \frac{s}{\rho}} = \sum_\rho\frac{1}{s -
        \rho}.
\end{equation*}
A similar argument holds for the series for $\Pi(s/2)$.


\bibliographystyle{apalike}

\bibliography{Complex_Analysis}

\end{document}
