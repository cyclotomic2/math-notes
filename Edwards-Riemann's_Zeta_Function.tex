\documentclass{note}

\usepackage{geometry}
\usepackage{amsmath}
\usepackage{xcolor}
\usepackage{fontspec}
\usepackage{titlesec}

\titleformat{\chapter}[display]
    {\LARGE\sffamily}
    {\chaptertitlename\space\thechapter}
    {1em}
    {\fontspec{Arial}\selectfont\bfseries\itshape}
\titleformat{\section}[block]
    {\Large\bfseries\filcenter}
    {\thesection}
    {1em}
    {\fontspec{Times New Roman}\selectfont\MakeUppercase}

\numberwithin{equation}{chapter}
\makeatletter
\def\tagform@#1{\maketag@@@{[\ignorespaces#1\unskip\@@italiccorr]}}
\makeatother

\renewcommand{\theenumi}{\alph{enumi}}

\newcommand{\piprod}[2]{
    \frac{1\cdot2\cdot\cdots\cdot #2}{\prn{#1+1}\prn{#1+2}\cdots\prn{#1+#2}}
    \prn{#2+1}^{#1}
}


\begin{document}

\chapter{Riemann's Paper}

\setcounter{section}{2}
\section{The Factorial Function}

\Page{8}

\begin{quotebar}
    On the other hand, it is not difficult to show [use formula (4) below] that the
    limit (3) exists for all values of $s$, real or complex, provided only that the
    denominator is not zero, that is, provided only that $s$ is not a negative
    integer.
\end{quotebar}

Rewrite the right-hand side of (4) as
\begin{equation}
    \prod_{n=1}^N \prn{1+\frac{s}{n}}^{-1} \prn{1+\frac{1}{n}}^{s} = \exp{ \sum_{n=1}^N
        \left[s\log\prn{1+\frac{1}{n}} - \log\prn{1+\frac{s}{n}} \right]}.
    \label{eq:rewrite-Pi-prod-form}
\end{equation}
Note the definition of $\log\prn{1+\frac{1}{n}}$ and $\log\prn{1+\frac{s}{n}}$
is unambiguous when $n$ is sufficiently large and $s$ is fixed. According to
Ahlfors [Theorem 8, p125], we can rewrite $\log(1+z)$ as
\begin{equation*}
    \log(1+z) = \log(1+0) + \frac{1}{1+0}z + f(z)z^2 = z + f(z)z^2
\end{equation*}
where $f$ is analytic in the domain of $\log(1+z)$ (chosen to be the complex
plain excluding real values $s \leq -1$). Thus
\begin{align}
    \abs{s\log\prn{1+\frac{1}{n}} - \log\prn{1+\frac{s}{n}}}
     & = \abs{ s\left[ \frac{1}{n} + f\prn{\frac{1}{n}}\prn{\frac{1}{n}}^2 \right] -
    \left[ \frac{s}{n} + f\prn{\frac{s}{n}}\prn{\frac{s}{n}}^2 \right] } \nonumber   \\
     & = \abs{ s f\prn{\frac{1}{n}} - s^2 f\prn{\frac{s}{n}} } \frac{1}{n^2}
    \label{eq:Pi-prod-summand-taylor}                                                \\
     & = O\prn{\frac{1}{n^2}} \quad \text{as } n \rightarrow \infty, \nonumber
\end{align}
which concludes that the sum on the right-hand side of
\eqref{eq:rewrite-Pi-prod-form} is absolute convergent.

\begin{quotebar}
    \begin{equation*}
        \Pi(s) = 2^s \Pi \prn{\frac{s}{2}} \Pi \prn{\frac{s - 1}{2}} \pi^{-1/2}
    \end{equation*}
\end{quotebar}

We will use the following fact.
\begin{lemma*}
    If both $\lim\limits_{n\rightarrow\infty} a_n$ and $\lim\limits_{n\rightarrow\infty}
        b_n$ exist, then $\lim\limits_{n\rightarrow\infty} a_n b_n$ exists and
    \begin{equation*}
        \lim_{n\rightarrow\infty} a_n b_n = \lim_{n\rightarrow\infty} a_n
        \lim_{n\rightarrow\infty} b_n.
    \end{equation*}
\end{lemma*}
\begin{proof}
    In fact, assume $\lim\limits_{n\rightarrow\infty} a_n = a$ and
    $\lim\limits_{n\rightarrow\infty} b_n = b$, then
    \begin{equation*}
        a_n b_n - ab = a_n b_n - a_n b + a_n b - ab = a_n(b_n - b) + b(a_n - a),
    \end{equation*}
    which approaches $0$ as $n \rightarrow \infty$.
\end{proof}
Now by equation (3) in the book,
\begin{align*}
    \Pi\prn{\frac{s}{2}}\Pi\prn{\frac{s-1}{2}}
     & = \lim_{N\rightarrow\infty} \left[ \piprod{\frac{s}{2}}{N} \right.             \\
     & \hspace{5em} \cdot \left. \piprod{\frac{s-1}{2}}{N} \right]                    \\
     & = \lim_{N\rightarrow\infty} \frac{\prn{1\cdot2\cdot\cdots\cdot N}^2}
    {\frac{1}{2^{2N}}\prn{s+1}\prn{s+2}\cdots\prn{s+2N}}\prn{N+1}^{s-\frac12}         \\
     & = \lim_{N\rightarrow\infty} \frac{2\cdot4\cdot\cdots\cdot 2N}{\prn{s+1}
        \prn{s+2}\cdots\prn{s+2N}} \cdot \frac{1\cdot2\cdot\cdots\cdot N}
    {\frac{1}{2^{N}}}\prn{N+1}^{s-\frac12}                                            \\
     & = \lim_{N\rightarrow\infty} \left[ \piprod{s}{2N} \right.                      \\
     & \hspace{5em} \left. \cdot \frac{1\cdot2\cdot\cdots\cdot N}{\frac{1}{2^{N}}
            1\cdot3\cdot\cdots\cdot (2N-1)} \cdot \frac{\prn{N+1}^{s-\frac12}}
    {\prn{2N+1}^s} \right]                                                            \\
     & = \lim_{N\rightarrow\infty} \left[ \piprod{s}{2N} \right.                      \\
     & \hspace{5em} \left. \cdot \piprod{-\frac12}{N} \cdot \frac{\prn{N+1}^{s}}
    {\prn{2N+1}^s} \right]                                                            \\
     & = \lim_{N\rightarrow\infty} \piprod{s}{2N}                                     \\
     & \hspace{5em} \cdot \lim_{N\rightarrow\infty} \piprod{-\frac12}{N}              \\
     & \hspace{5em} \cdot \lim_{N\rightarrow\infty}\frac{\prn{N+1}^{s}}{\prn{2N+1}^s} \\
     & = \Pi\prn{s} \Pi\prn{-\frac12} \prn{\frac12}^s = \Pi\prn{s} \pi^{\frac12}
    2^{-s},
\end{align*}
which implies
\begin{equation*}
    \Pi\prn{s} = 2^{s} \Pi\prn{\frac{s}{2}} \Pi\prn{\frac{s-1}{2}} \pi^{-\frac12}.
\end{equation*}

\begin{quotebar}
    Using it one can prove that $\Pi(s)$ is an analytic function of the complex
    variable s which has simple poles at $s = -1, -2, -3, \dots$.
\end{quotebar}

We showed above that $\Pi(s)$ is well-defined excluding $s = -1, -2, -3,
    \dots$. Now we show $\Pi(s)$ is analytic using equation
\eqref{eq:Pi-prod-summand-taylor}. According to Ahlfors [Theorem 1, p176], it
suffices to show that \eqref{eq:Pi-prod-summand-taylor} converges uniformly on
compact subsets of the domain of $\Pi(s)$, which is obvious by setting $n$
large enough and observing that $s$ is bounded on any compact subset (so that
$s/n$ is close to $0$ and the function $s f\prn{1/n} - s^2 f\prn{s/n}$ is
bounded).

To show that $\Pi(s)$ has simple poles at $s = -1, -2, -3, \dots$, it suffices
to multiply the product form of $\Pi(s)$ in \eqref{eq:rewrite-Pi-prod-form} by
$(s+n)$ and repeat the same argument to show $(s+n)\Pi(s)$ is analytic around
$s = -n$, for $n = 1, 2, 3, \dots$.

\end{document}
